\chapter{Manual}\label{ch:manual}


\ldots

\section{Tips and Tricks}


\subsection{Full cover page}

\textbf{Important}: most printing services will create their own cover page based
on the details you send them (title, name, affiliation, ...) and do not supply
you with all necessary parameters (e.g., thickness of the paper) because these
differ from machine to machine. Therefore, the generated cover page is only
indicative and probably not used by your printing server (or even correct).


A full cover page (combining front cover, spine and back cover) can be
generated automatically using the command \texttt{make cover} or \texttt{python3 run.py cover}. This creates a pdf
\texttt{\$(COVERPDF);} by default this is \texttt{cover.pdf}.

The width of the spine is set by redefining \texttt{\\adsphdspinewidth} (9mm by default).

It can be seen in the provided \texttt{thesis.tex} that all information necessary to
generate a cover page is contained between two markers

\begin{verbatim}
    %%% COVER: Settings %%%
    ...
    %%% COVER: End settings %%%
\end{verbatim}

DO NOT REMOVE THESE!! They are used by the Makefile!!

The default front and/or back cover page can be overwritten: 

\begin{itemize}
    \item create a file \texttt{mycoverpage.tex}
    \item redefine the commands \texttt{\\makefrontcovergeneral} and \texttt{\\makebackcovergeneral}. For
          an example and more information, see the provided file \texttt{mycoverpage.tex}.
\end{itemize}

The cover page in the generated pdf has the following structure:

{\tiny
\begin{verbatim}
    <--rbleed--><--backcoverpage--><--lbleed--><--spine width--><--lbleed--><--frontcoverpage--><--rbleed-->
\end{verbatim}
}

The default bleed (both lbleed and rbleed) is 7mm. I suggest not changing this
value unless you know what you are doing ;) The latter can be done by
redefining \texttt{\\defaultlbleed} and \texttt{\\defaultrbleed} respectively.


\subsection{Finetuning the backref package}

Finetuning the backref package (when \textbf{not} using biblatex). Add the
following to your preamble:

\begin{verbatim}
% See [http://n2.nabble.com/backref-td478438.html]:
% simple backref command redefinition to avoid double entries
\makeatletter
\ifadsphd@biblatex\else
  \ifadsphd@pagebackref%
    \renewcommand*{\backref}[1]{}
    % redefinition of the actually used \backrefalt 
    \renewcommand*{\backrefalt}[4]{% 
    \ifcase #1 % 
       % case: not cited 
    \or 
       % case: cited on exactly one page 
       Cited on page~#2.
    \else 
       % case: cited on multiple pages 
       Cited on pages~#2.
    \fi}
  \else
    \renewcommand*{\backref}[1]{}
    % redefinition of the actually used \backrefalt 
    \renewcommand*{\backrefalt}[4]{% 
    \ifcase #1 % 
       % case: not cited 
    \or 
       % case: cited on exactly one page 
       Cited in Section~#2.
    \else 
       % case: cited on multiple pages 
       Cited in Sections~#2.
    \fi}
  \fi
\fi
\makeatother
\end{verbatim}



%%%%%%%%%%%%%%%%%%%%%%%%%%%%%%%%%%%%%%%%%%%%%%%%%%
% Keep the following \cleardoublepage at the end of this file, 
% otherwise \includeonly includes empty pages.
\cleardoublepage

